Queremos agora resolver os problemas de controle ótimo até agora apresentados
numericamente, de forma que a solução seja uma aproximação a função $u^*$.
Para isso tomamos uma partição $\{t_0 = b_1, b_2, ..., b_{N+1} = t_1\}$,
usualmente com pontos igualmente espaçados, tal que a aproximação será dara
por $u_i \approx u(b_i)$. Sabemos que a solução deve satisfazer as condições
necessárias apresentadas no capítulo \ref{ch:1}. 

Em geral, a equação $\frac{\partial H}{\partial u} = 0$ pode ser manipulada
para encontrarmos $u$ em função de $x$ e de $\la$, para então termos a
expressão de $u^*$. A partir disso, podemos utilizar um método como
Runge-Kutta para resolver o sistema ótimo. Ele vai encontrar o controle ótimo
se esse existir. 

\section{Algoritmo}

O método apresentado a seguir é bem intuitivo e é conhecido como
\textit{Forward-Backward Sweep}. Seja $\vec{x} = (x_1,...,x_{N+1})$ e
$\vec{\la} = (\la_1, ..., \la_{N+1})$ vetores que aproximação nos pontos da
partição as funções estado e adjunta. \hl{Informa\c{c}\~oes sobre converg\^encia e
estabilidade podem ser encontradas em} \cite{wolfgang1978}. 

\begin{enumerate}
    \item Chute inicial para $\vec{u}$. 
    \item Usando a condição inicial $x(t_0)$ e os valores de $\vec{u}$,
    encontre $\vec{x}$ passo a frente através da equação diferencial. 
    \item Usando a condição de transversalidade $\la(t_1) = 0$ e os valores
    $\vec{u}$ e $\vec{x}$, resolva $\vec{\la}$ para trás de acordo com a
    equação adjunta. 
    \item Atualize o vetor de controle com os novos valores de $\vec{x}$ e
    $\vec{\la}$ através da equação $H_u = 0$. 
    \item Confira a convergência. Se dois passos subjacentes não estão
    suficientemente próximos, repita a partir do segundo passo. 
\end{enumerate}

Frequentemente é necessário usar uma combinação convexa \footnote{\textbf{Combinação Convexa:} Combinação linear de pontos, tal que os
coeficientes são não negativos e somam $1$.} entre dois controles
sequenciais para acelerar a convergência do algoritmo. Para os passos 2 e 3, o
método de Runge-Kutta é suficiente. 

Muitos tipos de teste de convergência existem. Frequentemente, considerar 
$$
||\vec{u} - \text{old }\vec{u}||_1 = \sum_{i=1}^{N+1} |u_i - \text{old }u_i| < \e   
$$
Nesse texto, usaremos o erro relativo com tolerância $\e$,
$$
\frac{||\vec{u} - \text{old }\vec{u}||_1}{||\vec{u}||_1} \le \e 
$$
Ou, de outra forma, queremos que 
$$
\e ||\vec{u}|| - ||\vec{u} - \text{old }\vec{u}|| \ge 0
$$
Vamos fazer esse requerimento para todas as variáveis, não apenas para o
controle. 

Ao longo do texto esse algoritmo será utilizado 

\hl{Especificar o c\'odigo utilizado, a linguagem e a organiza\c{c}\~ao dos
notebooks. Aqui ou no in\'icio do texto.}

\section{Runge-Kutta}

O método pode ser resumido pelas seguintes equações, dado um passo $h$. 
\begin{equation*}
    x(t + h) \approx x(t) + \frac{h}{6}(k_1 + 2k_2 + 2k_3 + k_4) 
\end{equation*}
\begin{equation*}
    \begin{cases}
    k1 = f(t,x(t)) \\
    k2 = f(t + \frac{h}{2},x(t) + \frac{h}{2}k_1) \\
    k3 = f(t + \frac{h}{2},x(t) + \frac{h}{2}k_2) \\
    k4 = f(t + h, x(t) + hk_3) \\
    \end{cases}
\end{equation*}
O erro é da ordem de $h^4$. 