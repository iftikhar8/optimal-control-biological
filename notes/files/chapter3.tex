\section{Termos Payoff}

Em muitos problemas, também queremos maximizar o valor de uma função em um
determinado ponto no tempo, como, por exemplo, no final do intervalo. Assim, o
problema é do tipo: 

\begin{equation*}
    \max_u \left[\phi(x(t_1)) + \int_{t_0}^{t_1} f(t, x(t),u(t))dt\right] 
\end{equation*}
\begin{equation*}
    x' = g(t,x(t),u(t)), x(t_0) = x_0
\end{equation*}

O termo $\phi(x(t_1))$ é conhecido como termo \textit{payoff}. Assim o
funcional objetivo se torna 
$$
J(u) = \int_{t_0}^{t_1} f(t,x(t),u(t))dt + \phi(x(t_1))
$$
A única mudança em relação ao cálculos do Capítulo \ref{ch:1} é a condição de transversalidade. 
$$\la(t_1) = \phi '(x^*(t_1)).$$ 

\section{Estados com Pontos Finais Fixos}

Considere o problema  
$$
\max_u \int_{t_0}^{t_1} f(t,x(t), u(t)) dt + \phi(x(t_0)) 
$$
$$
\text{sujeito a  }x'(t) = g(t,x(t),u(t)), x(t_1) = x_1
$$

Diferente do problema que estávamos estudando, fixamos o estado no ponto
final. Entretanto, o argumento utilizado no Capítulo \ref{ch:1} pode ser
replicado aqui. As condições necessárias serão as mesmas, exceto pela
condição de transversalidade. Especificamente, 
$$
\la(t_0) = \phi '(x(t_0))
$$
Isso sugere que exista uma dualidade entre as condições de fronteira do estado
e da adjunta. A maximização é sobre os controles \textit{admissíveis}, no
sentido que respeite todas as restrições definidas. 

Também podemos fixar os pontos inicial e final de estado. Notamos que estamos
considerando a maximização sobre o conjunto de controles admissíveis, que
respeitem as condições, inclusive sobre a variável de estado. Porém, nesse
caso, uma mudança nas condições necessárias é realizada no seguinte teorema. 

\begin{theorem}
    Se $u^*(t)$ e $x^*(t)$ são ótimos para o problema com pontos inicial e
    final fixados, então existe uma função $\la(t)$ diferenciável por partes e
    uma contante $\la_0$ igual a $0$ ou $1$, tal que 
    $$
    H(t,x^*(t),u(t),\la(t)) \le H(t,x^*(t),u^*(t), \la(t))
    $$
    para todos os controles admissíveis $u$ no tempo $t$ e o Hamiltoniano é 
    $$H = \la_0f(t,x(t),u(t)) + \la(t)g(t,x(t),u(t))$$ e
    $$\la '(t) = - \dfrac{\partial H(t,x^*(t), u^*(t))}{\partial x}.$$ 
\end{theorem}

A diferença das condições apresentadas no capítulo \ref{ch:1} é que a função adjunta não tem restrições. A demonstração utiliza
uma técnica diferente da utilizada até então \cite[147-153]{kamien2012dynamic}. A constante $\la_0$ ajusta para
problemas degenerados ou problemas onde o funcional objetivo é imaterial.

\begin{definition}
    O funcional objetivo ser imaterial significa que não depende da condição final do estado. 
\end{definition}

\section{Exemplos}

\begin{example}
    \begin{equation*}
        \min_u \frac{1}{2}\int_0^1 u(t)^2 dt + 5x(1)^2 
    \end{equation*}
    \begin{equation*}
        \text{sujeito a  }x'(t) = x(t) + u(t), x(0) = 1
    \end{equation*}
\end{example}

Observe que nesse exemplo estamos lidando com o termo \textit{payoff}
$5x(1)^2$, onde $\phi(x) = 5x^2$. Nesse caso
$$
H = \frac{1}{2}u^2 + \la(x + u)
$$
A condição de otimalidade, 
$$
0 = H_u = u^* + \la \implies u^*(t) = - \la(t)
$$
A equação adjunta, 
$$
\la '(t) = - H_x = - \la \implies \la(t) = Ce^{-t}
$$
A condição de transversalidade é 
$$
\la(1) = \phi '(x(1)) = 10x(1)
$$
Sabemos que $u^*(t) = -Ce^{-t}$. Usando a equação do estado, 
$$
x' = x - Ce^{-t} \implies x^{*}(t) = \frac{C}{2}e^{-t} + De^t
$$
Agora, utilizando as condições de fronteira,  
$$
\la(1) = Ce^{-1} = 10x^*(1) = 10\left(\frac{C}{2}e^{-1} + De\right)
$$
$$
x(0) = \frac{C}{2} + D = 1 \implies D = 1 - \frac{C}{2}
$$
Obtemos a equação
$$
e^{-1} = 5e^{-1} + 10e\frac{D}{C} \implies -\frac{4}{10}e^{-2} = \frac{D}{C} \implies 1 - \frac{C}{2} = -\frac{2}{5}Ce^{-2}
$$
Assim 
$$
C = \frac{1}{\frac{1}{2} - \frac{2}{5} e^{-2}} = \frac{10}{5 - 4e^{-2}} \implies D = 1 - \frac{5}{5 - 4e^{-2}} = \frac{-4e^{-2}}{5 - 4e^{-2}}
$$
Concluímos, portanto, que 
$$
u^{*}(t) = - \frac{10}{5 - 4e^{-2}}e^{-t} \text{ e }
x^{*}(t) = \frac{5}{5 - 4e^{-2}}e^{-t} - \frac{4e^{-2}}{5 - 4e^{-2}}e^t
$$

\begin{example}
    \begin{equation*}
        \min_u \int_0^1 u(t) dt
    \end{equation*}
    \begin{equation*}
        \text{sujeito a   }x'(t) = u(t)^2, x(0) = 0, x(1) = 0
    \end{equation*}
\end{example}

Observe que $x(t) = \int_0^t u(s)^2 ds$ e, como $x(1) = 0$, temos que $u
\equiv 0$ é o único controle admissível. Portanto ele será o único controle
ótimo. Agora, vamos examinar as condições necessárias, para fazer o
\textit{sanity check}. 
$$
H = \la_0 u + u^2 \la 
$$
Assim 
$$
0 = H_u = \la_0 + 2u \la \implies u^*(t) = -\frac{\la_0}{2\la(t)}
$$
Pela equação adjunta, $H_x = 0 \implies \la \equiv C$, para alguma constante
$C$. Isto é, $u^*(t) = -\la_0/2C$. Usando a equação do estado, obtemos que 
$$
x^*(t) = \la_0^2 \frac{t}{4C^2} + D
$$
tal que $D = 0$ e $\frac{\la_0^2}{4C^2} = 0 \implies \la_0 = 0$. Checamos
então que o Teorema é satisfeito com $\la_0 = 0$ e $u^* \equiv 0$, como já era
esperado. 

\begin{example}
    Seja $x(t)$ o número de célular de tumor no tempo $t$ com crescimento
    exponencial $\alpha$ e $u(t)$ a concentração de drogas. Queremos minimizar
    o número de células tumorais ao final do tratamento e os efeitos negativos
    acumulados do tratamento no corpo. Assim, o problema é resumido em 
    $$
    \min_u x(T) + \int_0^T u(t)^2 dt
    $$
    $$
    \text{sujeito a  }x'(t) = \alpha x(t) - u(t), x(0) = x_0 > 0
    $$
\end{example}

Esse é um simples modelo, não realístico, com objetivo ilustrativo apenas. O
termo \textit{payoff} é $X(T)$ e, portanto, $\phi(x) = x$. Podemos calcular as
condições necessárias. 
$$
H = u^2 + \la(\alpha x - u)
$$
$$
0 = H_u = 2u - \la \implies u^* = \frac{\la}{2}
$$
$$
\la ' = \dfrac{\partial H}{\partial x} = - \alpha \la \implies \la(t) = Ce^{-\alpha t}
$$
$$
\la(T) = \phi '(x(T)) = 1 \implies \la(t) = e^{\alpha(T - t)}
$$
Portanto o controle ótimo é $$u^{*}(t) = \frac{1}{2}e^{\alpha(T - t)},$$
Observando a equação do estado, temos que 
$$
x' - x = - \frac{1}{2}e^{\alpha(T - t)} \implies x^*(t) = x_0 e^{\alpha t} + e^{\alpha T}\frac{e^{-\alpha t} - e^{\alpha t}}{4\alpha}
$$
Com esse método, podemos obter a quantidade de droga a ser utilizada a cada
tempo $t$ e também saberemos a quantidade de células cancerosas. Todavia é
importante notar que esse é um modelo simplificado que não leva em
consideração diversos fatores importantes ao processo. 