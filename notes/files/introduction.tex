Nesse texto, estudam-se alguns problemas que envolvam encontrar um controle
ótimo, e toma como referência o livro de \cite{lenhart2007}. O texto terá
formato de notas com a mesma estrutura do livro e servirá
de guia introdutório sobre o tema na lingua portuguesa. Essa área da
matemática aplicada envolve estudos de otimização e de equações diferenciais e
pode ser ilustrada através de diversos exemplos e aplicações da ciência. 

Inicialmente, será apresentado um problema motivador que considera duas
equações diferenciais: uma representa a variação do peso da parte vegetativa
de uma planta, enquanto a outra representa o peso da parte reprodutiva. O
crescimento das plantas é modelado segundo o modelo de \cite{cohen1971299}.
Nesse caso, o \textit{controle} sobre o sistema é a fração da fotossíntese destinada para a parte vegetativa. Queremos \textit{maximizar} o 
crescimento da parte reprodutiva, que garante o mantimento da espécie. 

Sejam $x(t)$ a parte vegetativa e $y(t)$ a parte reprodutiva no
tempo $t$. Nosso objetivo será maximizar o funcional \refeq{funcional-maximize}
segundo a função $u(t)$ que representa a fração de fotossíntese para
o crescimento vegetativo: 

\begin{equation}
    \label{funcional-maximize}
    F(x,u,t) := \int_0^T \ln(y(t))dt, 
\end{equation}
onde $T$ é o limite superior do intervalo de tempo considerado e tal que o
modelo é um sistema de equações diferenciais com restrições: 
\begin{gather*}
    x'(t) = u(t)x(t) \\
    y'(t) = (1 - u(t))y(t) \\
    0 \leq u(t) \leq 1 \\
    x(0) > 0, \\
    y(0) \geq 0
\end{gather*}    

Um problema como esse é chamado de \textbf{problema de controle ótimo}, pois
queremos encontrar uma função $u$, denominada controle, ótima, segundo um
funcional objetivo. Nesse exemplo, podemos tirar conclusões interessantes
sobre o sistema, como, por exemplo, como a planta distribui seu fotossintato.
Outros problemas interessantes que surgem tem aplicações bem mundanas: qual a
porcentagem da população deveria ser vacinada em uma epidemia, a fim de que se
minimize o número de infectados e o custo de implementação? Qual a quantidade
de remédio deve ser ministrado para que se minimize a carga viral e a
quantidade administrada de remédio? Nesse caso a carga viral e a quantidade de
remédio formariam o sistema.  
Em um problemas como esse, encontramos: 
\begin{enumerate}
    \item variáveis de \textbf{estado}: descrevem a dinâmica do sistema. 
    \item variáveis de \textbf{controle}: conduzem o estado segundo uma ação. 
    \item \textbf{funcional} \footnote{Funcional: Mapa entre um conjunto de funções ao
    conjunto dos números reais} \textbf{objetivo}: Procuramos a função de
    controle de forma que esse funcional seja minimizado (ou maximizado). Ele
    representa o custo (ou ganho) ao se tomar uma atitude no sistema. 
\end{enumerate}

O texto terá como foco problemas que envolvam sistemas de equações
diferenciais ordinárias. Além disso, ao longo do texto, treze laboratórios são
desenvolvidos. Nesses laboratórios, uma aplicação biológica é estudada a fim
de apresentar conceitos dos capítulos que a precedem. Todos os laboratórios
estão em formato \texttt{jupyter-notebook} e escritos na linguagem de
programação \textit{Python}. A escolha dessa linguagem se deu a sua fácil
interpretação humana e interesse particular.  
