Nesse capítulo serão realizados os laboratórios 4, 5 e 6. Mais algumas
aplicações em biologia são desenvolvidas.

\begin{enumerate}[label=\textbf{Lab \arabic*:}]
    \setcounter{enumi}{3}

    \item Caso limitado 
    
    É um reexame do problema do primeiro laboratório. com a
    presença de limites inferior e superior para o controle. O objetivo é
    mostrar como usar a classe desenvolvida em \textit{Python} para esse tipo
    de problema.

    \item Câncer
    
    Modelo do tratamento de células cancerosas por quimioterapia, com objetivo
    de minimizar os efeitos negativos do uso das drogas, mas também minimizar
    a densidade dessas células no corpo. É simplificado, porém apresenta dois
    componentes importantes: o crescimento Gompertzian e a hipótese de Skipper
    para a morte das células segundo o uso das drogas. 

    \item Colheita de peixe  
    
    Uma população de peixes é inserida em um tanque e é deixada para ser
    caçada, além da morte natural, mas sem taxa de nascimento. Queremos maximizar
    a massa de peixes caçada, enquanto minimizamos o gasto com a colheita.
    Restrições são consideradas. 
 
\end{enumerate}