Nesse capítulo serão realizados os laboratórios 11, 12 e 13. Mais algumas
aplicações em biologia são desenvolvidas.

\begin{enumerate}[label=\textbf{Lab \arabic*:}]
    \setcounter{enumi}{10}

    \item Extração de Madeira
    
    Uma fazenda de extração de madeira é considerada e produz madeira. Essa
    quantidade de madeira é vendida para o mercado que paga um valor. Esse
    valor pode ser reinvestido na fazenda em terras e trabalho ou pode ser
    dado como lucro e, então, investido com uma taxa de juros. A decisão do
    fazendeiro é o quanto reinvestir e quando reinvestir do dinheiro ganhado.
    Esse é um problema Bang Bang em que se utiliza o algoritmo
    \textit{Forward-Backward Sweep} desenvolvido até então. 

    \item Reator Biológico 
    
    Algumas bactérias são capazes de degradar o contaminante em solos
    contaminados. Um método para limpar essas áreas é, portanto, favorecer o
    aumento do nível de bactérias através da injeção de nutrientes
    necessárias para o metabolismo e crescimento da colônia. Nesse laboratório
    vamos abordar uma simplificação desse problema em um sistema controlado,
    como um reator biológico. 

    \item Modelo Predador Presa 
    
    Modelo simples com predador e presa com uma restrição isoperimétrica.
    Queremos estudar a situação onde a presa é uma peste que deve ser
    combatida por um pesticida, que servirá de controle. Mas esse pesticida
    também afeta o predador, o qual não queremos que aconteça. Esse modelo
    permite o estudo do método estudado no capítulo \ref{ch:21}. 
    

\end{enumerate}