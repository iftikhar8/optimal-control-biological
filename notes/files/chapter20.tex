Em alguns problemas reais, não temos um tempo fixo para procurar um controle
ótimo, mas também queremos encontrar o tempo ótimo que devemos permanecer com
ele, como, por exemplo, o tempo para um tratamento. Precisamos refazer as
condições necessárias, dada que temos mais variáveis desconhecidas.  

\section{Condições necessárias}

Sejam $f$ e $g$ funções continuamente diferenciáveis e considere o problema 

$$
\max_{u,T} J(u,T) = \max_{u,T} \int_{t_0}^T f(t, x(t), u(t))dt + \phi(T, x(T))
$$
$$
\text{sujeito a   }x'(t) = g(t, x(t), u(t)), x(t_0) = x_0.
$$

Seja $(u^*, T^*)$ um par ótimo, de forma que $u^*$ esteja definida em $[t_0,
T^*]$ e $J(u, T) \le J(u^*, T^*) < \infty$, para todos os controles $u$ e
tempos $T$. Seja $x^*$ o estado correspondente. Tome $h$ uma função contínua
por partes e $\e \in \mathbb{R}$, de forma que $u^{\e}(t) = u^*(t) + \e h(t)$
é um controle. Assim 
\begin{equation*}
    0 = \lim_{\e \to 0} \frac{J(u^*, T^*) - J(u^{\e}, T^*)}{\e}.
\end{equation*}
Pelo mesmo argumento dos Capítulo \ref{ch:1} e \ref{ch:3}, temos que 
\begin{gather*}
    0 = H_u \text{ em } u^* \\
    \la ' = - H_x = -f_x - \la g_x \\
    \la(T^*) = \phi_x(T^*, x(T^*)),
\end{gather*}
Mas isso não nos dá informação sobre $T^*$. Por isso, seja $\delta \le t_0 -
T^*$, tal que $T^* + \delta$ seja um tempo admissível. Primeiro, assuma que
$u^*$ é contínua à esquerda em $T^*$. Se não o for, mude seu valor nesse ponto
se necessário. Então, seja $u^*(t) = u^*(T), t > T^*$. Portanto, 
\begin{equation*}
    0 = \lim_{\e \to 0} \frac{J(u^*, T^* + \delta) - J(u^*, T^*)}{\delta}.
\end{equation*}
Equivalentemente, dado que $u^*$ é contínua em $T^*$ e $x^*$ é diferenciável,
usamos o Teorema Fundamental do Cálculo e a regra do produto para obter
\begin{equation*}
    \begin{split}
        0 &= \lim_{\delta \to 0} \frac{1}{\delta}\int_{T^*}^{T^* + \delta} f(t, x^*, u^*) dt \\ 
        &~~~~~~~~+ \frac{\phi(T^* + \delta, x^*(T^* + \delta)) - \phi(T^*, x^*(T^*))}{\delta} \\
        &= f(T^*, x^*(T^*), u^*(T^*)) + \phi_t(T^*, x^*(T^*)) + \phi_x(T^*, x^*(T^*))\frac{dx^*}{dt}(T^*) \\
        &=  f(T^*, x^*(T^*), u^*(T^*)) + \phi_t(T^*, x^*(T^*)) + \la(T^*)g(T^*, x^*(T^*), u^*(T^*)) \\
        &= H(T^*, x^*(T^*), u^*(T^*), \la(T^*)) + \phi_t(T^*, x^*(T^*))
    \end{split}
\end{equation*}

Deveria ficar claro que as mesmas condições necessárias vem com a presença de
limites nos controles, ou quando existem mais de uma variável de estado e
controle. O que não fica claro, observando as contas já feitas no Capítulo
\ref{ch:3}, é como esse problema é afetado quando os pontos inicial e final
são fixados. De fato, as mesmas condições necessárias aparecem. 

\section{Controle ótimo temporal}

A ideia é a seguinte: mover as variáveis de estado de uma posição inicial para
uma posição final específica em um período mínimo de tempo. Como isso parece
se relacionar com nosso problema? 

Note que $T = \int_0^T 1 dt$. Então, podemos reformular o problema como 
\begin{gather*}
    \min_{u,T} \int_0^T 1 dt, \\ 
    \text{sujeito a   }x'(t) = g(t, x(t), u(t)), x(0) = x_0, x(T) = x_1, \\
    a \le u(t) \le b, 
\end{gather*}

\section{Exemplos}

\begin{example}
    \begin{gather*}
        \min_{u,T} x(T) + \int_0^T u(t)^2 dt \\
        \text{sujeito a   }x'(t) = \alpha x(t) -u(t), x(0) = x_0, \alpha > 0.
    \end{gather*}
\end{example}

Podemos formar as condições necessárias 
\begin{gather*}
    H = u^2 + \alpha x \la - u \la, \\
    0 = H_u = 2u - \la \implies u^* = \la/2, \\
    \la ' = -H_x = -\alpha \la \implies \la(t) = Ke^{-\alpha t}
    \la(T^*) = \phi_x(x^*(T^*)) = 1, 
\end{gather*}

Usando essas condições, concluímos que 
\begin{gather*}
    \la(t) = e^{\alpha(T^* - t)} \\
    u^*(t) = \frac{1}{2}e^{\alpha(T^* - t)} \\
    x^*(t) = x_0e^{\alpha t} + e^{\alpha T^*}\frac{e^{-\alpha t} - e^{\alpha t}}{4\alpha}
\end{gather*}

Por fim, a condição sobre $T^*$, dado que $\phi_t = 0$,
\begin{equation*}
    \begin{split}
        0 &= H(T^*, x^*(T^*), u^*(T^*), \la(T^*)) \\
        &= \frac{1}{4} + \alpha x_0e^{\alpha T^*} + \frac{1 -e^{2\alpha T^*}}{4} - \frac{1}{2} \\
        &= \alpha x_0e^{\alpha T^*} - \frac{e^{2\alpha T^*}}{4}
    \end{split}
\end{equation*}
Assim $\log(4\alpha x_0) + \alpha T^*= 2 \alpha T^* \implies  T^* =
\frac{1}{\alpha}\log{4\alpha x_0}$. Isso faz sentido se $\alpha x_0 > 1/4$. 

\begin{example}
    \begin{gather*}
        \min_{u,T} \int_0^T 1 dt \\ 
        \text{sujeito a   }x'(t) = x(t)u(t)  - \frac{1}{2}u(t)^2, x(0) = x_0 \in (0,1), x(T) = 1.
    \end{gather*}
\end{example}

Escrevemos o Hamiltoniano $H = 1 + xu\lambda - \frac{1}{2}u^2\lambda$. Assim,
escrevemos as condições necessárias, 

$$
\la ' = -H_x = - \la u \implies \la(t)\exp\left(-\int_0^t u(s)ds \right), 
$$
para alguma constante $C$. Se $C = 0 \implies H \equiv 1$ o que contradiz o
Hamiltoniano ser $0$ em $T^*$. Portanto $C \neq 0$. A condição de otimalidade
nos dá que 
$$
0 = H_u = \la(x - u) \implies u^* = x^*
$$
Fazendo a substituição na equação de estado, obtemos que 
$$
x' = \frac{1}{2}x^2 \implies x^*(t) = \frac{2x_0}{2 - x_0 t} = u^*(t)
$$
A condição $x(T^*) = 1 \implies T^* = 2/x_0 - 2$.  