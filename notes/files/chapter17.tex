Nesse capítulo vamos considerar problemas em que o Hamiltoniano é linear em
$u$ e, portanto, não podemos obter uma caracterização para $u$ através de $H_u
= 0$. Em geral, a solução ótima envolve descontinuidades em $u^*$.  

\section{Controles Bang-Bang}

Considere o problema de controle ótimo.
\begin{gather*}
    \max_u \int_{t_0}^{t_1} f_1(t,x) + u(t)f_2(t,x) dt \\ 
    \text{sujeito a  }x'(t) = g_1(t,x) + u(t)g_2(t,x), x(0) = x_0 \\
    a \leq u(t) \leq b 
\end{gather*}

Podemos escrever o Hamiltoniano como 
$$H(t,x,u,\la) = f_1(t,x) + \la(t) g_1(t,x) + u(t)(f_2(t,x) + \la(t) g_2(t,x)).$$

A condição de otimalidade dada por
$$
\frac{\partial H}{\partial u} = f_2(t,x) + \la(t)g_2(t,x), 
$$
não carrega informação sobre $u(t)$. Assim definimos 
$$\psi(t) := f_2(t,x(t)) + \la(t)g_2(t,x(t)),$$ muitas vezes chamada de \textit{função de troca}.  A caracterização
de $u^*$ é 
$$
u^*(t) = \begin{cases} 
    a &\text{se } \psi(t) < 0, \\
    ? &\text{se } \psi(t) = 0, \\
    b &\text{se } \psi(t) > 0. 
\end{cases}
$$
Se $\psi = 0$ não pode ser mantido em um intervalo de tempo, mas ocorre apenas
em uma quantidade finita de pontos, o controle é dito \textbf{Bang Bang},
porque só varia entre os valores mínimo e máximo de $u(t)$, bastando obter os
pontos de troca. Os valores de $u(t)$ nesses pontos não são de interesse,
portanto. Se $\psi(t) \equiv 0$ em um intervalo de tempo, dizemos que $u^*$ é \textbf{singular} nesse intervalo. Esse caso será explorado na próxima sessão. 

O método \textit{forward-backward sweep} pode ser empregado nesse caso.
Primeiro precisamos provar que o problema é de fato bang-bang, isto é $\psi =
0$ não ocorre em um intervalo.  Caso isso seja verdadeiro, basta
fazermos, em código \textit{Python}, 

\begin{verbatim}
    u = lambda t,x,adjoint: a if psi(t, x, adjoint) < 0 else b
\end{verbatim}

\begin{example}
    \begin{gather*}
        \max_u \int_0^2 e^t(1 - u(t))dt \\ 
        \text{sujeito a  }x'(t) = u(t)x(t), x(0) = 1 \\
        0 \le u(t) \le 1.
    \end{gather*}

    Esperamos que $u \equiv 0$ para maximizar a expressão. 
\end{example}

O Hamiltoniano é 
$$
H = e^t(1 - u) + \la u x
$$
A equação adjunta e a condição de transversalidade são 
$$
\la '(t) = - H_x = - \la(t)u(t), \la(2) = 0,
$$
e
$$
\psi(t) = H_u = -e^t + \la(t)x(t)
$$

Suponha que $\psi \equiv 0$ em um intervalo $I \subset (0,2)$. Portanto,
diferenciando a expressão $e^t = \la(t)x(t)$, 
$$
e^t = \la '(t)x(t) + \la(t) x'(t) = -\la(t)u(t)x(t) + \la(t)u(t)x(t) = 0,
$$
o que é impossível. Portanto provamos que o problema é bang-bang. 
$$
u^* = 0 \implies x' = \la ' = 0 \implies x, \la \text{ constantes.}
$$
$$
u^* = 1\implies \la' = -\la \implies \la(t) = Ce^{-t}
$$
Agora temos que determinar os intervalos. Como $\la(2) = 0$, qualquer um dos
casos acima implicaria $\la(t) \equiv 0$ em algum intervalo que inclua $t=2$. 
Se $u^* = 1$ nesse intervalo, teremos que $C = 0$ e, para garantir a
continuidade de $\la$ em $[0,2]$, teremos que $\la \equiv 0$. Se $u^* = 0$
nesse intervalo, seja $I = (a,b) \subset [0,2)$ o maior intervalo tal que $u^* =
1$. Como $\la(b) = 0 \implies C = 0$ a única possibilidade que garante
continuidade. Concluímos que $\la \equiv 0$. Nesse sentido 
$$
\psi(t) = -e^t < 0 \implies u^* \equiv 0 \text{ e }x^* \equiv 1
$$

\section{Controles singulares}

Consideramos dois exemplos simples para motivar o tópico. 

\begin{example}
    \begin{gather*}
        \max_u \int_0^2 (x(t) - t^2)^2 dt \\
        \text{sujeito a  }x'(t) = u(t), x(0) = 1 \\
        0 \leq u(t) \leq 4
    \end{gather*}
\end{example}

Primeiro vamos calcular as condições necessárias, 
$$
H = (x - t^2)^2 + \la u 
$$
$$
\la '(t) = -H_x = -2(x - t^2), \la(2) = 0
$$
$$
\psi(t) = H_u = \la(t)
$$
Se $\psi \equiv 0$ em algum intervalo, então 
$$
0 \equiv \la '(t) = -2(x - t^2) \equiv x(t) =  t^2,
$$
então nesse intervalo $u = x' = 2t$.
Consequentemente, 
$$
u^*(t) = \begin{cases}
    0 &\text{quando }\la > 0, \\
    2t &\text{quando }\la = 0, \\
    4 &\text{quando }\la < 0.
\end{cases}
$$

Vamos provar que $x^*(t) \ge t^2$ no intervalo $[0,2]$ na Proposição
\ref{prop1}. Com isso demonstrado, temos que $\la ' \le 0$ em $[0,2]$. Como
$\la(2) = 0$, devemos ter que $\la \ge 0$ em $[0,2]$, pois a função decresce
até $0$. Portanto existe $k \in [0,2]$ tal que $\la > 0$ em $[0,k)$ e $\la = 0$ em $[k,2]$. 

Suponha que $k = 0$. Então $\la \equiv \la ' \equiv 0$. Mas $x^*(0) > 0^2$,
então $\la '(0) < 0$, contradição. 

Suponha que $k = 2$. Então $u^* \equiv 0 \implies x^* \equiv 1$. Isso
contradiz $x^*(2) \ge 2^2$. Portanto $0 < k < 2$ e 
\begin{equation}
    \label{eq1:u-star}
    u^*(t) = \begin{cases}
        0 &\text{se }0 \le t < k, \\
        2t &\text{se }k < t \le 2, 
    \end{cases}\text{  }x^*(t) = \begin{cases}
        1 &\text{se }0 \le t < k, \\
        t^2 + (1 - k^2) &\text{se }k \le t < 2,
    \end{cases}    
\end{equation}
dado que $x^*$ é contínua. Precisamos encontrar $k$. 

Note que $\la \equiv 0$ em $[k,2]$. Assim, nesse intervalo, 
$$
0 = \la '(t) = - 2(1 - k^2) \implies k = 1.
$$

\begin{proposition}
    \label{prop1}
    $$x^*(t) \ge t^2, \forall t \in [0,2]$$
\end{proposition}

\begin{proof}
    Suponha que $x(t) < t^2$ para algum $t$. Como $x(t) > t^2$ em $t=0$,
    existe $t_0 \in (0,2)$ tal que $x(t_0) \le t_0^2$ e $u(t_0) = x'(t_0) <
    2t_0 < 4$. Portanto $u(t_0) = 0$ e $\la(t_0) > 0$. 

    Seja $t_1 := \inf\{t: \la(t) = 0, t \in (t_0, 2]\}$. Como esse conjunto é
    fechado (tome uma sequência convergente nele e veja que o limite precisa estar
    nele também), o valor de $t_1$ pertence a ele, isto é, $\la(t_1) = 0, t_1 > t_0$.
    Portanto $\la(t) > 0$ e $u^*(t) = 0$ quando $t \in [t_0,
    t_1)$. Assim $x^*(t) = x^*(t_0), t \in [t_0, t_1)$. Como $x^*(t_0) \le
    t_0^2$, então $x^*(t) \le t^2$. Pela equação adjunta, $\la '(t) \ge 0$ em
    $[t_0, t_1)$. Mas se $\la(t_0) > 0$, $\la(t_1) = 0$ é impossível. Com essa
    contradição concluímos que $x^*(t) \ge t^2, t \in [0,2]$.   
\end{proof}

Se não fôssemos capazes de encontrar os intervalos onde $u$ é definido,
precisaríamos utilizar um método numérico sobre \refeq{eq1:u-star}. Porém,
numericamente, não é possível obter $\la = 0$. Para isso, poderíamos
considerar que $u^*(t) = 2t$ quanto $|\la| < \e = 0.00001$, por exemplo.
Entretanto, problemas singulares tendem a ser instáveis. 

Pesquisadores a partir de comportamentos de controles ótimos singulares,
desenvolveram condições necessárias adicionais. A mais notável é a condição
generalizada de Legendre-Clebsh (\cite{Tsypkin1970AppliedOC}, \cite{Grossmann1984}, \cite{krener1977}). É uma condição de \textit{segunda ordem} e
envolve derivadas de ordem mais alta do Hamiltoniano. 

\begin{example}
    Reservas marinhas proibidas são tema de discussão entre aqueles que
    enfatizam os benefícios de conservação e aqueles que enfatizam a redução
    da produção pesqueira. \cite{neubert2003} investigou o papel dessas reservas para
    maximar o rendimento segundo uma estratégia de caça. 

    A equação diferencial parcial da densidade de estoque relativa a
    capacidade de carga $w(x,t)$ é 
    $$ 
    w_t(x,t) = w_{xx}(x,t) + w(x,t)(1 - w(x,t)) - u(x)w(x,t),
    $$
    Suponha que a densidade é estável no tempo. Assim, 
    $$
    0 = w''(x) + w(x)(1 - w(x)) - u(x)w(x),
    $$
    tal que $w(0) = w(l) = 0$, isto é, as regiões de fronteira são inabitadas.
    Assumimos que $0 \le u(x) \le 1$ é o controle de colheita. 

    Primeiro convertemos a equação em um sistema de primeira ordem:
    \begin{gather*}
        w'(x) = v(x), w(0) = w(l) = 0, \\
        v'(x) = u(x)w(x) - w(x)(1 - w(x)), 
    \end{gather*}
    Queremos maximizar o retorno em $[0,l]$.
    $$
    J(u) = \int_0^l u(x)w(x) dx
    $$
    Vamos ter a restrição adicional que $w(x) > 0$ no interior do domínio. 
\end{example}

Primeiro vamos encontrar as condições necessárias do problema. 
\begin{gather*}
    H = uw + \la_1 v + \la_2(uw - w(1 - w)), \\
    \la_1 ' = -H_w = -u - \la_2(u - 1 + 2w), \\
    \la_2 ' = -H_v = - \la_1, \la_2(0) = \la_2(l) = 0, \\
    \psi = H_u = w(1 + \la_2)
\end{gather*}

Se $\psi \equiv 0$ em algum intervalo, temos que $\la_2 \equiv -1 \implies
\la_2' = 0 \implies \la_1 = 0$ nesse intervalo. Assim 
$$
-u + u - 1 + 2w = 0 \implies w^* = \frac{1}{2}
$$
Assim $w' = v = v' = 0 = u^* - (1 - w^*)$ e $u^* = \frac{1}{2}$. Concluímos
que 
\begin{equation}
    u^*(t) = \begin{cases}
        1 &\text{se }\la_2 > -1, \\
        1/2 &\text{se }\la_2 = -1, \\
        0 &\text{se }\la_2 < -1
    \end{cases}
\end{equation}

As condições necessárias devem ser resolvidas numericamente. A partir de
resultados numéricos, existe pelo menos um intervalo onde o controle ótimo é
zero, o que significa uma região marítima proibida. 